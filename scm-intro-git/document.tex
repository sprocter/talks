%% LaTeX Beamer presentation template (requires beamer package)
%% see http://bitbucket.org/rivanvx/beamer/wiki/Home
%% idea contributed by H. Turgut Uyar
%% template based on a template by Till Tantau
%% this template is still evolving - it might differ in future releases!

\documentclass{beamer}

\mode<presentation>
{
\usetheme{Warsaw}

\setbeamercovered{transparent}
}

\usepackage[english]{babel}
\usepackage[latin1]{inputenc}

% font definitions, try \usepackage{ae} instead of the following
% three lines if you don't like this look
\usepackage{mathptmx}
\usepackage[scaled=.90]{helvet}
\usepackage{courier}
\usepackage{tabulary}


\usepackage[T1]{fontenc}


\title{Introduction to source control with git}

%\subtitle{}

% - Use the \inst{?} command only if the authors have different
%   affiliation.
\author{Sam Procter}
% \author{\inst{1}}

% - Use the \inst command only if there are several affiliations.
% - Keep it simple, no one is interested in your street address.
\institute{Department of Computing and Information Sciences\\
Kansas State University}

\date{Tuesday, September 16, 2014}


% This is only inserted into the PDF information catalog. Can be left
% out.
\subject{Talks}



% If you have a file called "university-logo-filename.xxx", where xxx
% is a graphic format that can be processed by latex or pdflatex,
% resp., then you can add a logo as follows:

% \pgfdeclareimage[height=0.5cm]{university-logo}{university-logo-filename}
% \logo{\pgfuseimage{university-logo}}



% Delete this, if you do not want the table of contents to pop up at
% the beginning of each subsection:
\AtBeginSubsection[]
{
\begin{frame}<beamer>
\frametitle{Outline}
\tableofcontents[currentsection,currentsubsection]
\end{frame}
}

% If you wish to uncover everything in a step-wise fashion, uncomment
% the following command:

%\beamerdefaultoverlayspecification{<+->}

\begin{document}

\begin{frame}
\titlepage
\end{frame}

\begin{frame}
\frametitle{Outline}
\tableofcontents
% You might wish to add the option [pausesections]
\end{frame}


\section{Source Control Management}

\subsection[Why Source Control?]{Why use source control?}

\begin{frame}
\frametitle{``Hello''}
\framesubtitle{Subtitles are optional}

Cuz it's da bes'

\end{frame}

\subsection[Source Control History]{A history of Source Control Tools}

\begin{frame}
% \frametitle{``Hello''}
% \framesubtitle{Subtitles are optional}

\begin{table}
\scriptsize
\centering
\begin{tabulary}{\textwidth}{l|l|l|L|L}
Gen. & Networking & Operations & Concurrency & Examples \\\hline
1 & None & One file at a time & Locks & SCCS\\
2 & Centralized & Multi-file & Merge before commit & CVS, SVN\\
3 & Distributed & Changesets & Commit before merge & git, hg
\end{tabulary}
\caption{Adapted from \cite{sink:BOOK2011}}
\end{table}
\begin{enumerate}
  \uncover<1->{\item RCS: No sharing }
  \uncover<2->{\item Second item.}
\end{enumerate}
\end{frame}

% \begin{frame}
% \frametitle{}
% 
% % You can create overlays
% \begin{itemize}
%   \item using the \texttt{pause} command:
%   \begin{itemize}
%     \item First item.
%     \pause
%     \item Second item.
%   \end{itemize}
%   \item using overlay specifications:
%   \begin{itemize}
%     \item<3-> First item.
%     \item<4-> Second item.
%   \end{itemize}
%   \item using the general \texttt{uncover} command:
%   \begin{itemize}
%     \uncover<5->{\item First item.}
%     \uncover<6->{\item Second item.}
%   \end{itemize}
% \end{itemize}
% \end{frame}

\section{git}
\subsection[Why Source Control?]{Why use source control?}

\section*{Be Brave!}

% \section*{Summary}
% 
% \begin{frame}
% \frametitle<presentation>{Summary}
% 
% \begin{itemize}
%   \item The \alert{first main message} of your talk in one or two lines.
% \end{itemize}
% 
% % The following outlook is optional.
% \vskip0pt plus.5fill
% \begin{itemize}
%   \item Outlook
%   \begin{itemize}
%     \item Something you haven't solved.
%     \item Something else you haven't solved.
%   \end{itemize}
% \end{itemize}
% \end{frame}

\begin{frame}[allowframebreaks]
  \frametitle<presentation>{References}    
  \begin{thebibliography}{10}    
  \beamertemplatebookbibitems
  \bibitem{sink:BOOK2011}
    Eric Sink.
    \newblock {\em Version control by example}.
    \newblock Pyrenean Gold Press, 2011.
  \bibitem{sink:BOOK2011}
    Scott Chacon.
    \newblock {\em Pro Git}.
    \newblock Apress, 2009.
%   \beamertemplatearticlebibitems
%   \bibitem{Jemand2000}
%     S.~Jemand.
%     \newblock On this and that.
%     \newblock {\em Journal of This and That}, 2(1):50--100, 2000.
  \end{thebibliography}
\end{frame}
\end{document}
